\section{Fourier}
en esta seccion se realizo el analisis de dos ondas senosoidales en el cual la primera señal estaba conformado por la suma de dos señales y en el segundo caso se encontraban a destiempo es decir primero se generaba una y luego la otra

\begin{figure}[h!]
\centering
\includegraphics[scale=0.5]{RonderosCarlos_signalSumasub.pdf}
\caption{grafica signal.dat y signalsuma.dat}
\label{Fig.}
\end{figure}

posteriormente se realizo la implementacion propia de la transformada de fourier para despues ser utilizada para obtener la transformaada de los datos de signal.dat y signalsuma.dat

\begin{figure}[h!]
\centering
\includegraphics[scale=0.5]{RonderosCarlos_transformada.pdf}
\caption{Transformada signal.dat }
\label{Fig.}
\end{figure}

\begin{figure}[h!]
\centering
\includegraphics[scale=0.5]{RonderosCarlos_transformada_originalSuma.pdf}
\caption{Transformada signalsuma.dat }
\label{Fig.}
\end{figure}

En esta seccion tambien se le realizo un espectrograma a los valores obtenidos por la implementacion de la transformada de fourier a los dos grupos de datos de signal.dat y signalsuma.dat

\begin{figure}[h!]
\centering
\includegraphics[scale=0.5]{espectrograma_original.pdf}
\caption{Espectrograma de Transformada signal.dat }
\label{Fig.}
\end{figure}

\begin{figure}[h!]
\centering
\includegraphics[scale=0.5]{espectrograma_originalSumado.pdf}
\caption{Espectrograma de Transformada signalSuma.dat }
\label{Fig.}
\end{figure}

para terminar con esta seccion se cargaron los datos de temblor.txt en donde se realizo un agrafica de la señal vs el tiempo ademas de un espectrograma 

\begin{figure}[h!]
\centering
\includegraphics[scale=0.5]{espectrograma_temblor.pdf}
\caption{grafico señal vs t y Espectrograma de temblor.txt }
\label{Fig.}
\end{figure}

ademas se realizo la implemntacion de la transformada de fourier con los paquetes de scipy a los datos provenientes de temblor.txt

\begin{figure}[h!]
\centering
\includegraphics[scale=0.5]{transformada_temblor.pdf}
\caption{Transformada temblor.txt }
\label{Fig.}
\end{figure}


\section{ODE}

\begin{figure}[h!]
\centering
\includegraphics[scale=0.5]{amplitud_maxima.pdf}
\caption{Transformada temblor.txt }
\label{Fig.}
\end{figure}

\begin{figure}[h!]
\centering
\includegraphics[scale=0.5]{plot1.pdf}
\caption{Transformada temblor.txt }
\label{Fig.}
\end{figure}

\begin{figure}[h!]
\centering
\includegraphics[scale=0.5]{plot2.pdf}
\caption{Transformada temblor.txt }
\label{Fig.}
\end{figure}

\begin{figure}[h!]
\centering
\includegraphics[scale=0.5]{plot3.pdf}
\caption{Transformada temblor.txt }
\label{Fig.}
\end{figure}

\begin{figure}[h!]
\centering
\includegraphics[scale=0.5]{plot4.pdf}
\caption{Transformada temblor.txt }
\label{Fig.}
\end{figure}


\end{document}
