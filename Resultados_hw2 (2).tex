
\documentclass[12pt]{article}
\usepackage{graphicx}
\author{Carlos Ronderos}
\title{Tarea 2, 2019-01 Metodos Computacionales}

\begin{document}
\maketitle

\section{Fourier}

En esta sección se realizó el análisis de dos ondas sinusoidales en la cual la primera señal estaba conformada por la suma de dos señales y en el segundo caso se encontraba un grupo de señales que se generaban en diferentes tiempos es decir una de las señales se generaba con anterioridad a la otra.

\begin{figure}[h!]
\centering
\includegraphics[scale=0.5]{RonderosCarlos_signalSumasub.pdf}
\caption{grafica signal.dat y signalsuma.dat}
\label{Fig.}
\end{figure}

Por otra parte, se realizó la implementación propia de la transformada de Fourier en la cual realice una corrección con el manejo de suma para que no me arrojara un error de tipo Warning, para posteriormente ser utilizada en la obtención de la transformada de Fourier de los datos signal.dat y signalsuma.dat.

\begin{figure}[h!]
\centering
\includegraphics[scale=0.5]{RonderosCarlos_transformada.pdf}
\caption{Transformada signal.dat }
\label{Fig.}
\end{figure}

\begin{figure}[h!]
\centering
\includegraphics[scale=0.5]{RonderosCarlos_transformada_originalSuma.pdf}
\caption{Transformada signalsuma.dat }
\label{Fig.}
\end{figure}

De las gráficas fue posible apreciar que de signalSuma.dat se presentaron 4 picos de frecuencia con mismos valores en su amplitud mientras  que de los datos de signal.dat se obtuvieron 4 picos pero con diferentes amplitudes ya que los valores obtenidos en los extremos son mayores a los picos intermedios.
Por otra parte, se realizaron los respectivos espectrogramas de las transformadas de los datos signal.dat y signalSuma.dat

\begin{figure}[h!]
\centering
\includegraphics[scale=0.5]{espectrograma_original.pdf}
\caption{Espectrograma de Transformada signal.dat }
\label{Fig.}
\end{figure}

\begin{figure}[h!]
\centering
\includegraphics[scale=0.5]{espectrograma_originalSumado.pdf}
\caption{Espectrograma de Transformada signalSuma.dat }
\label{Fig.}
\end{figure}

Para terminar con esta sección se cargaron los datos de temblor.txt en donde se realizó una gráfica en forma de subplot donde se puede apreciar los datos de temblor.txt vs t en donde se logra observar el punto en que sucede el temblor. Seguidamente se puede apreciar el espectrograma del temblor en el que se puede ver que la región antes del movimiento se aprecia de un color diferente al pos-temblor en el que además se ve una curva decayendo con una coloración diferente (amarilla).
  
\begin{figure}[h!]
\centering
\includegraphics[scale=0.5]{espectrograma_temblor.pdf}
\caption{grafico señal vs t y Espectrograma de temblor.txt }
\label{Fig.}
\end{figure}

Ademas, se realizo la transformada de fourier con los paquetes de scipy a los datos de temblor.txt en donde se pueden aprecair 3 picos de frecuencia dos de los cuales tienen una intensidad menor que el valor central en 0.0 

\begin{figure}[h!]
\centering
\includegraphics[scale=0.5]{transformada_temblor.pdf}
\caption{Transformada temblor.txt }
\label{Fig.}
\end{figure}


\section{Edificios ODE}

Después de realizar el tratamiento de los datos y obtención de los diferentes casos en los que se variaba la frecuencia de forzamiento se obtuvo que la gráfica para la amplitud máxima en función de la frecuencia es:  

\begin{figure}[h!]
\centering
\includegraphics[scale=0.7]{amplitud_maxima.pdf}
\caption{Amplitud máxima }
\label{Fig.}
\end{figure}

En esta se aprecia que la máxima amplitud se logra en la frecuencia de forzamiento se 0.65 además se observó la presencia de dos picos adicionales los cuales disminuían su amplitud en función de la frecuencia  

\begin{figure}[h!]
\centering
\includegraphics[scale=0.6]{plot1.pdf}
\caption{frecuencia 0.65 resonancia }
\label{Fig.}
\end{figure}
la gráfica 1 corresponde al valor de frecuencia 0.65 la cual concuerda con la frecuencia de resonancia de nuestro sistema por tal razón los bloques se mueven simétricamente. 
\begin{figure}[h!]
\centering
\includegraphics[scale=0.5]{plot2.pdf}
\caption{Frecuencia 1.3 caos }
\label{Fig.}
\end{figure}

La gráfica 2 corresponde al valor de frecuencia 1.3, en la gráfica se muestra el efecto que se genera en los bloques al moverse independientemente, ya que estos no perciben la fuerza al mismo tiempo. Por tal razón, los bloques se mueven caóticamente sin aumentar su amplitud. 

\begin{figure}[h!]
\centering
\includegraphics[scale=0.5]{plot3.pdf}
\caption{Frecuencia 2.5 simetria  }
\label{Fig.}
\end{figure}

en la gráfica 3 corresponde al valor de frencuencia de 2.5 en la grafica se muestran valores periodicos para este sistema este valor es proximo a uno de los picos de amplitud mas especificamente en el pico con menor valor de amplitud de la grafica de maximas amplitudes.

\begin{figure}[h!]
\centering
\includegraphics[scale=0.5]{plot4.pdf}
\caption{Frecuencia 4 caos }
\label{Fig.}
\end{figure}
La gráfica 4 corresponde al valor de frecuencia 4, en la gráfica se muestra el efecto que se genera en los bloques al moverse independientemente, ya que estos no perciben la fuerza al mismo tiempo. Por tal razón, los bloques se mueven caóticamente sin aumentar su amplitud. 

\end{document}
